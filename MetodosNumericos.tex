\documentclass[12pt,a4paper,twoside]{article}
\usepackage[utf8]{inputenc}
\usepackage[spanish]{babel}
\usepackage{amsmath}
\usepackage{amsfonts}
\usepackage{amssymb}
\usepackage{graphicx}
\author{Cristian}
\title{Métodos Numéricos \\
\large Teoremas, propiedades y más}
\makeindex

\newcounter{DefCounter}
\setcounter{DefCounter}{0}
\newcounter{PropCounter}
\setcounter{PropCounter}{0}

\newcommand{\definicion}[1]{
\stepcounter{DefCounter}

\textbf{\underline{\scalebox{1.1}{Definición \arabic{DefCounter}}}}
\addcontentsline{toc}{subsection}{Definición \arabic{DefCounter}}

\vspace{10pt}

#1

\vspace{20pt}


}

\newcommand{\prop}[1]{
\stepcounter{PropCounter}

\textbf{\underline{\scalebox{1.1}{Propiedad \arabic{PropCounter}}}}
\addcontentsline{toc}{subsection}{Propiedad \arabic{PropCounter}}

\vspace{10pt}

#1

\vspace{20pt}


}

\newcommand{\propDem}[2]{
\stepcounter{PropCounter}

\textbf{\underline{\scalebox{1.1}{Propiedad \arabic{PropCounter}}}}
\addcontentsline{toc}{subsection}{Propiedad \arabic{PropCounter}}

\vspace{10pt}

#1


\vspace{10pt}

\textbf{\underline{\scalebox{1.1}{Dem}}}

\vspace{10pt}

#2

\vspace{20pt}

}

\begin{document}
\maketitle
\cleardoublepage
\tableofcontents
\cleardoublepage
\section{Disclaimer}
Este documento fue hecho como resumen de teoremas, propiedades, etc. para la materia \textit{Metodos Númericos} de la carrera \textit{Ciencias de la Computación} de la \textit{Facultad de Ciencias Exactas y Naturales}. De ninguna manera pretende remplazar las clases ni asegura estar completo/correcto. A su vez, las propiedades y sus demostraciones están basadas en mis apuntes de las clases al menos que se indique lo contrario. Esto quiere decir que puede no coincidir con demostraciones de otras fuentes o de otras cursadas.
\\
En caso de encontrar algún posible error se recomienda verificarlo con un docente de la materia y en caso de efectivamente serlo, subirlo como issue indicando como se confirmó que lo era (por ejemplo, docente que lo confirmo o libro).
\cleardoublepage
\section{Practica 1}
\subsection{Definiciones, propiedades,etc.}
\definicion{
$v_0,\ldots , v_k \in \mathbb{R}^{n}$ son linialmente independientes si

$$\sum_{i=0}^{k} \alpha_i v_i = 0 \Rightarrow \alpha_i = 0 \quad \forall i = 0 \ldots k$$
}

\definicion{
Una matriz es inversible $\Leftrightarrow$ su determinante es distinto de cero.
}

\definicion{
Traza de A es $\sum_{i}^{n}a_{ii}$
}

\definicion{
$A \in \mathbb{R}^{m \times n}$ se dice triangular superior (t.s.) si $a_{ij} = 0 \quad \forall i > j$
}

\definicion{
$A \in \mathbb{R}^{m \times n}$ se dice triangular inferior (t.i.) si $a_{ij} = 0 \quad \forall i < j$
}

\prop{
Producto de t.s. da t.s.
}

\prop{
Producto de t.i. da t.i.
}

\prop{
$(AB)^{t} = B^{t}A^{t}$
}

\prop{
Determinante
\begin{itemize}
\item $A$ inversible $\Leftrightarrow$ $det(A)\neq 0$
\item $det(AB) = det(A)det(B)$
\item Sea $A$ triangular, $det(A) = \prod_{i=1}^{n}a_{ii}$
\item $det(A) = det(A^{t})$
\item $det(\alpha A) = \alpha^{n}det(A)$
\item $det(A^{-1}) = \frac{1}{det(A)}$
\end{itemize}
}

\definicion{
Nucleo de $A \in \mathbb{R}^{m \times n}$
$$Nu(A)=\left\lbrace x \in \mathbb{R}^{n} / Ax=0 \in \mathbb{R}^{m} \right\rbrace$$
}

\definicion{
Imagen de $A \in \mathbb{R}^{m \times n}$
$$Im(A)=\left\lbrace y \in \mathbb{R}^{m} / \exists x \in \mathbb{R}^{n} : Ax=y \right\rbrace$$
}

\prop{
Teorema de la dimensión
$A \in \mathbb{R}^{m \times n}$
$$dim(Nu(A)) + dim(Im(A)) = n$$
}

\definicion{
\begin{itemize}
\item Rango fila de $A$ es cantidad de filas l.i.
\item Rango columna de $A$ es cantidad de columnas l.i.
\end{itemize}
}

\prop{
Sea $A \in \mathbb{R}^{n \times n}$, son equivalentes
\begin{itemize}
\item $A$ inversible.
\item $\nexists x \in \mathbb{R}^{n}, x \neq 0$, tal que $Ax=0$
\item Las columnas de $A$ son l.i.
\item Las filas de $A$ son l.i.
\end{itemize}
}
\footnote{ej 21 práctica 1}

\prop{
Sean $A,B \in \mathbb{R}^{n \times n}$ inversibles
\begin{itemize}
\item $(A^{-1})^{-1} = A$
\item $(AB)^{-1} = B^{-1}A^{-1}$
\item $(A^{t})^{-1} = (A^{-1})^{t}$
\item Si $A$ t.i. $\Rightarrow$ $A^{-1}$ es t.i.
\end{itemize}
\footnote{ej 23 práctica 1}
}

\definicion{
Norma vectorial\\
$\| . \| : \mathbb{R}^{n} \rightarrow \mathbb{R}$, $x, y \in \mathbb{R}^{n}, \alpha \in \mathbb{R}$ define una norma vectorial si
\begin{itemize}
\item $\| x \| \geq 0$ y $\| x \| = 0 \Leftrightarrow x=\vec{0}$
\item $\| x + y \| \leq \| x \| + \| y \|$
\item $\| \alpha x \| = |\alpha| \| x \|$
\end{itemize}
}

\definicion{
$\| x \|_{p} = \sqrt[p]{\sum |x_{i}|^{p}}$
}
\definicion{
$\| x \|_{\infty} = max_{1 \leq i \leq n}(|x_{i}|)$
}

\subsection{Ejercicios}
\cleardoublepage
\section{Practica 2}
\subsection{Definiciones, propiedades,etc.}
\subsection{Ejercicios}

\cleardoublepage
\section{Practica 3}
\subsection{Definiciones, propiedades,etc.}
\definicion{$A \in \mathbb{R}^{n \times n}$ se dice simetrica definida positiva (s.d.p.) si
\\
\begin{center}
$A$ es simétrica.
\end{center}
\begin{center}
$x^{t}Ax > 0$ $\forall x \neq \vec{0}$ con $x \in \mathbb{R}^{n}$
\end{center}
}

\propDem{
$A$ sdp $\Rightarrow$ $A$ es inversible.
}
{
Supongamos $A$ no inversible. $\exists$ $x^{*} \neq \vec{0} / Ax^{*}=\vec{0} \Rightarrow {x^{*}}^{t}Ax^{*}=0$ Abs!
}

\prop{
Sea $A$ simétrica. DP $\Leftrightarrow$ las submatrices principales son no singulares.
\footnote{$\Rightarrow$ de teórica. $\Leftarrow$ ejercicio 9 de práctica 3}
}

\prop{
Sea $A$ sdp $\Rightarrow$ $A=LU$
}

\prop{
Sea $A$ sdp $\Leftrightarrow$ $A=LL^{t}$ (\textit{factorización de Cholesky}) donde $L$ es triangular inferior con no necesariamente $1$s en la diagonal.
\footnote{$\Rightarrow$ de teórica. $\Leftarrow$ de ejercicio 8 de práctica 3 que dice
\\
Si $A=LL^{t}$ es una factorización de $A$ con $L$ una matriz triangular inferior con elementos de la diagonal positivos, $A$ es sdp.
}
}

\prop{
Sea $A$ sdp. $\forall x$, $y \in \mathbb{R}^{n}$
\begin{itemize}
\item Si $x$ e $y$ son l.i. $\vert x^{t}Ay \vert < \sqrt{x^{t}Ax}\sqrt{y^{t}Ay}$
\item Si $x$ e $y$ son l.d. $\vert x^{t}Ay \vert = \sqrt{x^{t}Ax}\sqrt{y^{t}Ay}$
\end{itemize}
\footnote{Ejercicio 4 de práctica 3}
}

\subsection{Ejercicios}

\end{document}