\documentclass[10pt,a4paper,twoside]{article}
\usepackage[utf8]{inputenc}
\usepackage[spanish]{babel}
\usepackage{amsmath}
\usepackage{amsfonts}
\usepackage{amssymb}
\author{Cristian}
\title{Métodos Numéricos \\
\large Teoremas, propiedades y más}
\makeindex

\newcommand{\definicion}[1]{\textbf{Definición:} #1}
\newcommand{\prop}[1]{\textbf{Propiedad:} #1}
\newcommand{\propDem}[2]{\textbf{Propiedad:} #1 \\ \textbf{Dem:} #2}

\begin{document}
\maketitle
\cleardoublepage
\section{Disclaimer}
Este documento fue hecho como resumen de teoremas, propiedades, etc. para la materia \textit{Metodos Númericos} de la carrera \textit{Ciencias de la Computación} de la \textit{Facultad de Ciencias Exactas y Naturales}. De ninguna manera pretende remplazar las clases ni asegura estar completo/correcto. A su vez, las propiedades y sus demostraciones están basadas en mis apuntes de las clases al menos que se indique lo contrario. Esto quiere decir que puede no coincidir con demostraciones de otras fuentes o de otras cursadas.
\\
En caso de encontrar algún posible error se recomienda verificarlo con un docente de la materia y en caso de efectivamente serlo, subirlo como issue indicando como se confirmó que lo era (por ejemplo, docente que lo confirmo o libro).
\cleardoublepage
\section{Practica 1}
\cleardoublepage
\section{Practica 2}
\cleardoublepage
\section{Practica 3}
\definicion{$A \in \mathbb{R}^{n \times n}$ se dice simetrica definida positiva (s.d.p.) si
\\
\begin{center}
$A$ es simétrica.
\end{center}
\begin{center}
$x^{t}Ax > 0$ $\forall x \neq \overrightarrow{0}$ con $x \in \mathbb{R}^{n}$
\end{center}
}
\end{document}