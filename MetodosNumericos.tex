\documentclass[12pt,a4paper,twoside]{article}
\usepackage[utf8]{inputenc}
\usepackage[spanish]{babel}
\usepackage{amsmath}
\usepackage{amsfonts}
\usepackage{amssymb}
\usepackage{graphicx}
\author{Cristian}
\title{Métodos Numéricos \\
\large Teoremas, propiedades y más}
\makeindex

\newcounter{DefCounter}
\setcounter{DefCounter}{0}
\newcounter{PropCounter}
\setcounter{PropCounter}{0}

\newcommand{\definicion}[1]{
\stepcounter{DefCounter}

\textbf{\underline{\scalebox{1.1}{Definición \arabic{DefCounter}}}}
\addcontentsline{toc}{subsection}{Definición \arabic{DefCounter}}

\vspace{10pt}

#1

\vspace{20pt}


}

\newcommand{\prop}[1]{
\stepcounter{PropCounter}

\textbf{\underline{\scalebox{1.1}{Propiedad \arabic{PropCounter}}}}
\addcontentsline{toc}{subsection}{Propiedad \arabic{PropCounter}}

\vspace{10pt}

#1

\vspace{20pt}


}

\newcommand{\propDem}[2]{
\stepcounter{PropCounter}

\textbf{\underline{\scalebox{1.1}{Propiedad \arabic{PropCounter}}}}
\addcontentsline{toc}{subsection}{Propiedad \arabic{PropCounter}}

\vspace{10pt}

#1


\vspace{10pt}

\textbf{\underline{\scalebox{1.1}{Dem}}}

\vspace{10pt}

#2

\vspace{20pt}

}

\begin{document}
\maketitle
\cleardoublepage
\tableofcontents
\cleardoublepage
\section{Disclaimer}
Este documento fue hecho como resumen de teoremas, propiedades, etc. para la materia \textit{Metodos Númericos} de la carrera \textit{Ciencias de la Computación} de la \textit{Facultad de Ciencias Exactas y Naturales}. De ninguna manera pretende remplazar las clases ni asegura estar completo/correcto. A su vez, las propiedades y sus demostraciones están basadas en mis apuntes de las clases al menos que se indique lo contrario. Esto quiere decir que puede no coincidir con demostraciones de otras fuentes o de otras cursadas.
\\
En caso de encontrar algún posible error se recomienda verificarlo con un docente de la materia y en caso de efectivamente serlo, subirlo como issue indicando como se confirmó que lo era (por ejemplo, docente que lo confirmo o libro).
\cleardoublepage
\section{Practica 1}
\cleardoublepage
\section{Practica 2}
\cleardoublepage
\section{Practica 3}
\definicion{$A \in \mathbb{R}^{n \times n}$ se dice simetrica definida positiva (s.d.p.) si
\\
\begin{center}
$A$ es simétrica.
\end{center}
\begin{center}
$x^{t}Ax > 0$ $\forall x \neq \vec{0}$ con $x \in \mathbb{R}^{n}$
\end{center}
}

\propDem{
$A$ sdp $\Rightarrow$ $A$ es inversible.
}
{
Supongamos $A$ no inversible. $\exists$ $x^{*} \neq \vec{0} / Ax^{*}=\vec{0} \Rightarrow {x^{*}}^{t}Ax^{*}=0$ Abs!
}

\prop{
Sea $A$ simétrica. DP $\Leftrightarrow$ las submatrices principales son no singulares.
\footnote{$\Rightarrow$ de teórica. $\Leftarrow$ ejercicio 9 de práctica 3}
}

\prop{
Sea $A$ sdp $\Rightarrow$ $A=LU$
}

\prop{
Sea $A$ sdp $\Leftrightarrow$ $A=LL^{t}$ (\textit{factorización de Cholesky}) donde $L$ es triangular inferior con no necesariamente $1$s en la diagonal.
\footnote{$\Rightarrow$ de teórica. $\Leftarrow$ de ejercicio 8 de práctica 3 que dice
\\
Si $A=LL^{t}$ es una factorización de $A$ con $L$ una matriz triangular inferior con elementos de la diagonal positivos, $A$ es sdp.
}
}

\prop{
Sea $A$ sdp. $\forall x$, $y \in \mathbb{R}^{n}$
\begin{itemize}
\item Si $x$ e $y$ son l.i. $\vert x^{t}Ay \vert < \sqrt{x^{t}Ax}\sqrt{y^{t}Ay}$
\item Si $x$ e $y$ son l.d. $\vert x^{t}Ay \vert = \sqrt{x^{t}Ax}\sqrt{y^{t}Ay}$
\end{itemize}
\footnote{Ejercicio 4 de práctica 3}
}
\end{document}